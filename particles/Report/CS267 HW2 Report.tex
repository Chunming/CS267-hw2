\documentclass[11pt]{amsart}
\usepackage{geometry}                % See geometry.pdf to learn the layout options. There are lots.
\geometry{letterpaper}                   % ... or a4paper or a5paper or ... 
%\geometry{landscape}                % Activate for for rotated page geometry
%\usepackage[parfill]{parskip}    % Activate to begin paragraphs with an empty line rather than an indent
\usepackage{graphicx}
\usepackage{amssymb}
\usepackage{epstopdf}
\DeclareGraphicsRule{.tif}{png}{.png}{`convert #1 `dirname #1`/`basename #1 .tif`.png}

\title{CS267 Assignment 2: Parallelize Particle}
\author{Chun Ming Chin, Chris Melgaard, Viraj Kulkarni }
%\date{}                                           % Activate to display a given date or no date

\begin{document}
\maketitle
%\section{}
%\subsection{}



The box is segmented into subBlockNum number of subBlocks. Each subBlock is represented as a bin in the binArray.

The binArray is a 2D static array data structure, where we allocate binNum number of elements to represent the number of subBlocks in the box. 

Each binArray[idx], where 0 <= idx < binNum, is a pointer to an array of particlet pointers. Dereferencing binArray[idx][jdx], where 0<=jdx<maxN gives a particlet object. We determine the upper bound, maxN using a mathematical argument. 

To check if a target particle collides with its neighboring particles, we only check a subset of all n particles. This subset comprises the following: 
\begin{enumerate}
\item Particles that belong to the same bin as the target particle
\item Particles that belong to the left, right, bottom, top, topLeft, topRight, bottomLeft or bottomRight subBlocks with respect to the original subBlock where the target particle is located. 
\end{enumerate}


\section{Pseudocode for serial.cpp O(n) implementation}

Initialize particle binning

for each time step {

   // Compute forces for each particle in each bin
   
   for each bin {
   
      for each particle in current bin {
         Compare with other particles in current subBlock.
         Compare with other particles in adjacent 8 subBlocks.
      }    
   }
   
   // Move particles
   
   // Re-bin particles
   
}


\end{document}  